\documentclass[12pt]{article}
\usepackage{amsmath}
\usepackage{graphicx}
\usepackage{hyperref}
\usepackage[latin1]{inputenc}

\begin{document}
Execise 8

Theorem: Prove (from the definition of a limit of a sequence) that if the
sequence $\{a_n\}_{n=1}^{\infty}$ tends to limit $L$ as
$n\rightarrow\infty$, then for any fixed number $M>0$ the sequence
$\{Ma_n\}_{n=1}^{\infty}$ tends to the limit $ML$.

Proof: We proof this theorem with the definition of a limit of a sequence. A sequence converges to a limit, if the following is true:

\[\forall n(|x_n-L| < \frac{\epsilon}{|M|}), where\]
\[\epsilon \in \mathbb{R}, \epsilon > 0, n \in \mathbb{N}, N \in \mathbb{N}, n \geq N\]

Following this we get:
\[|Mx_n-ML| < M\frac{\epsilon}{|M|}\]
\[|M||x_n-L| < \epsilon\]

Hence, the sequence converges to ML.
\end{document}


\
\\
\\

Execise 9

Theorem: Given an infinite collection $A_n,n=1,2,...$ of intervals of the real line, their intersection is defined to be $\cap_{n=1}^{\infty}A_n=\{x|(\forall n)(x\in A_n)\}$. Give an example of a family of intervals $A_n,n=1,2,...$ such that $A_{n+1}\subset A_n$ for all $n$ and $\cap_{n=1}^{\infty}A_n=\O$. Prove that your example has the stated property.

Proof: A family of intervalls with the mentioned properties are the following:
\[A_n=(0,\frac{1}{n}), n = 1,2,...\]
We proof the the first property by taking an $n \in N$.Then every element of those two intervall $(0,\frac{1}{n})$ and $(0,\frac{1}{n+1})$ are between 0 and $\frac{1}{n}$, respectively  $\frac{1}{n+1}$. Following that $\frac{1}{n+1} < \frac{1}{n}$, $A_{n+1}$ must be a subset of $A_n$.
Now we have to proof the second property. As we learned from the first proof that the collection $A_{n+1}$ is always a subset of $A_n$, we can conclude that all $A_n$ are an intersection of $A_\infty$ when $n \to \infty$.
Since $\lim \Limits_{n \to \infty} \frac{1}{n} = 0$, the intervall $(0, 0)$ is an empty set, which proofs the second property. 

\
\\
\\

Execise 10

Theorem: Give an example of a family of intervals $A_n,n=1,2,...$ such that $A_{n+1}\subset A_n$ for all $n$ and $\cap_{n=1}^{\infty}A_n$ consists of a single real number. Prove that your example has the stated property.


Proof: This proof is similar to theorem 9 but with an closed intervall. A family of intervalls with the mentioned properties are the following:
\[A_n=[0,\frac{1}{n}], n = 1,2,...\]
We proof the the first property by taking an $n \in N$.Then every element of those two intervall $[0,\frac{1}{n}]$ and $[0,\frac{1}{n+1}]$ are $0 \leq x \leq$ $\frac{1}{n}$, respectively  $\frac{1}{n+1}$. Following that $\frac{1}{n+1} < \frac{1}{n}$, $A_{n+1}$ must be a subset of $A_n$.
Now we have to proof the second property. As we learned from the first proof that the collection $A_{n+1}$ is always a subset of $A_n$, we can conclude that all $A_n$ are an intersection of $A_\infty$ when $n \to \infty$.
Since $\lim \Limits_{n \to \infty} \frac{1}{n} = 0$, the intervall $[0, 0]$ is a set, which consists only of a singular real number $0$.



