\documentclass[12pt]{article}
\usepackage{amsmath}
\usepackage{graphicx}
\usepackage{hyperref}
\usepackage[latin1]{inputenc}

\begin{document}
Execise 4

Theorem: Every odd number is one of the form $4n + 1$ or $4n + 3$.

Proof: We proof this withn the division theorem
The division theorem states, that every natural number can be expressed as $n = ab + r, with\: a, r \in N and\: b \in Z$ and $0<=r<a$. Since $a = 4$ there are four possible cases, that describes any natural number. These are:
\[4b,\: 4b+1,\: 4b+2,\: 4b+3\]
Since any natural numbers of the form $4b$ or $4b+2$ always even due to the factor 4. Hence, the only cases that a odd natural number can be expressed in is $4b+1$ or $4b+3$, which are the forms in the theorem.
\end{document}
Execise 5

Theorem: For any integer n, at least one of $n$, $n+2$ or $n+4$ is divisble by 3.

Proof: We proof this with the division theorem. The division theorem states, that every natural number can be expressed as $n = ab + r, with\: a, r \in N and\: b \in Z$ and $0<=r<a$. Since $a = 3$ there are three possible cases, that describes any natural number.

\[3b,\: 3b + 1, \: 3b + 2\]

First case $(n = 3b)$:
\[3/3b\:\lor\:3/3b+2\:\lor\:3/3b+4\]
\[True\:\lor\:False\:\lor\:False = True\]

Second case $(n = 3b+1)$:
\[3/3b+1\:\lor\:3/3b+1+2\:\lor\:3/3b+1+4\]
\[False\:\lor\:True\:\lor\:False = True\]

Third case $(n = 3b+2)$:
\[3/3b+1\:\lor\:3/3b+2+2\:\lor\:3/3b+2+4\]
\[False\:\lor\:False\:\lor\:True = True\]

Since for every of the third cases one of $n$, $n+2$, $n+4$ is true. Hence the theorem is true.


\
\\

Execise 6

Theorem: A classic unsolved problem in number theory asks if there are infinitely many pairs of "twin primes", pairs of primes separated by 2, such as 3 and 5, 11 and 13, or 71 and 73. Prove that the only prime triple (i.e. three primes, each 2 from the next) is 3, 5, 7.

Proof: This theorem is partly proofed by assignment five. It states that for any integer n at least one of $n$, $n+2$, $n+4$ is divisble by 3. But since a prime is only divisble by himself or by 1, there can´t exist any triple prime, other than $3,5,7$.


\
\\

Execise 7

Theorem: Prove that for any natural number $2+2^2+2^3+...2^n = 2^{n+1} -2$

Proof: We proof this theorem by induction. 
\[A(n) = \sum_{i=1}^{n}(2^i), B(n) = 2^{n+1}-2\]
\[A(1) = B(1)\]
\[2^1 = 2^{1+1} -2 \]
\[2 = 2 \]
The theorem is true for 1. Furthermore it must be true for $n+1$ to be true.

\[A(n+1) = B(n+1)\]
\[\sum_{i=1}^{n+1}(2^i) = 2^{n+2}-2 \]
Since we have a geometric series at the left hand side, we bring the sum into closed form first.
\[\frac{2(1-2^{n+2})}{1-2} = 2^{n+2}-2 \]
\[2^{n+2}-2 = 2^{n+2}-2, by\: Algebra \]

Hence the theorem is true. 



\
\\
\\

Execise 8

Theorem: Prove (from the definition of a limit of a sequence) that if the
sequence $\{a_n\}_{n=1}^{\infty}$ tends to limit $L$ as
$n\rightarrow\infty$, then for any fixed number $M>0$ the sequence
$\{Ma_n\}_{n=1}^{\infty}$ tends to the limit $ML$.

Proof: We proof this theorem with the definition of a limit of a sequence. A sequence converges to a limit, if the following is true:

\[\forall n(|x_n-L| < \frac{\epsilon}{|M|}), where\]
\[\epsilon \in \mathbb{R}, \epsilon > 0, n \in \mathbb{N}, N \in \mathbb{N}, n \geq N\]

Following this we get:
\[|Mx_n-ML| < M\frac{\epsilon}{|M|}\]
\[|M||x_n-L| < \epsilon\]

Hence, the sequence converges to ML.


\
\\
\\

Execise 9

Theorem: Given an infinite collection $A_n,n=1,2,...$ of intervals of the real line, their intersection is defined to be $\cap_{n=1}^{\infty}A_n=\{x|(\forall n)(x\in A_n)\}$. Give an example of a family of intervals $A_n,n=1,2,...$ such that $A_{n+1}\subset A_n$ for all $n$ and $\cap_{n=1}^{\infty}A_n=\O$. Prove that your example has the stated property.

Proof: A family of intervalls with the mentioned properties are the following:
\[A_n=(0,\frac{1}{n}), n = 1,2,...\]
We proof the the first property by taking an $n \in N$.Then every element of those two intervall $(0,\frac{1}{n})$ and $(0,\frac{1}{n+1})$ are between 0 and $\frac{1}{n}$, respectively  $\frac{1}{n+1}$. Following that $\frac{1}{n+1} < \frac{1}{n}$, $A_{n+1}$ must be a subset of $A_n$.
Now we have to proof the second property. As we learned from the first proof that the collection $A_{n+1}$ is always a subset of $A_n$, we can conclude that all $A_n$ are an intersection of $A_\infty$ when $n \to \infty$.
Since $\lim \Limits_{n \to \infty} \frac{1}{n} = 0$, the intervall $(0, 0)$ is an empty set, which proofs the second property. 

\
\\
\\

Execise 10

Theorem: Give an example of a family of intervals $A_n,n=1,2,...$ such that $A_{n+1}\subset A_n$ for all $n$ and $\cap_{n=1}^{\infty}A_n$ consists of a single real number. Prove that your example has the stated property.


Proof: This proof is similar to theorem 9 but with an closed intervall. A family of intervalls with the mentioned properties are the following:
\[A_n=[0,\frac{1}{n}], n = 1,2,...\]
We proof the the first property by taking an $n \in N$.Then every element of those two intervall $[0,\frac{1}{n}]$ and $[0,\frac{1}{n+1}]$ are $0 \leq x \leq$ $\frac{1}{n}$, respectively  $\frac{1}{n+1}$. Following that $\frac{1}{n+1} < \frac{1}{n}$, $A_{n+1}$ must be a subset of $A_n$.
Now we have to proof the second property. As we learned from the first proof that the collection $A_{n+1}$ is always a subset of $A_n$, we can conclude that all $A_n$ are an intersection of $A_\infty$ when $n \to \infty$.
Since $\lim \Limits_{n \to \infty} \frac{1}{n} = 0$, the intervall $[0, 0]$ is a set, which consists only of a singular real number $0$.



